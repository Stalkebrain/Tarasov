\documentclass[11pt]{article}
\usepackage{amsmath,amssymb,amsthm}
\usepackage{algorithm}
\usepackage[noend]{algpseudocode} 

%---enable russian----

\usepackage[utf8]{inputenc}
\usepackage[russian]{babel}


% PROBABILITY SYMBOLS
\newcommand*\PROB\Pr 
\DeclareMathOperator*{\EXPECT}{\mathbb{E}}


% Sets, Rngs, ets 
\newcommand{\N}{{{\mathbb N}}}
\newcommand{\Z}{{{\mathbb Z}}}
\newcommand{\R}{{{\mathbb R}}}
\newcommand{\Zp}{\ints_p} % Integers modulo p
\newcommand{\Zq}{\ints_q} % Integers modulo q
\newcommand{\Zn}{\ints_N} % Integers modulo N

% Landau 
\newcommand{\bigO}{\mathcal{O}}
\newcommand*{\OLandau}{\bigO}
\newcommand*{\WLandau}{\Omega}
\newcommand*{\xOLandau}{\widetilde{\OLandau}}
\newcommand*{\xWLandau}{\widetilde{\WLandau}}
\newcommand*{\TLandau}{\Theta}
\newcommand*{\xTLandau}{\widetilde{\TLandau}}
\newcommand{\smallo}{o} %technically, an omicron
\newcommand{\softO}{\widetilde{\bigO}}
\newcommand{\wLandau}{\omega}
\newcommand{\negl}{\mathrm{negl}} 

% Misc
\newcommand{\eps}{\varepsilon}
\newcommand{\inprod}[1]{\left\langle #1 \right\rangle}


\newcommand{\handout}[5]{
	\noindent
	\begin{center}
		\framebox{
			\vbox{
				\hbox to 5.78in { {\bf Научно-исследовательская практика} \hfill #2 }
				\vspace{4mm}
				\hbox to 5.78in { {\Large \hfill #5  \hfill} }
				\vspace{2mm}
				\hbox to 5.78in { {\em #3 \hfill #4} }
			}
		}
	\end{center}
	\vspace*{4mm}
}

\newcommand{\lecture}[4]{\handout{#1}{#2}{#3}{Scribe: #4}{Теоретико - числовые функции #1}}

\newtheorem{theorem}{Теорема}
\newtheorem{lemma}{Лемма}
\newtheorem{definition}{Определение}
\newtheorem{corollary}{Следствие}
\newtheorem{fact}{Факт}

% 1-inch margins
\topmargin 0pt
\advance \topmargin by -\headheight
\advance \topmargin by -\headsep
\textheight 9.2in
\oddsidemargin 0pt
\evensidemargin \oddsidemargin
\marginparwidth 0.5in
\textwidth 6.5in


\parindent 0.6in
\parskip 0ex

\begin{document}
	\lecture{}{Лето 2020}{}{Тарасов Егор}
	\begin{corollary}	
	Функции $\tau$ и $\sigma$ являются мультипликативными функциями.
    \end{corollary}

    \begin{proof}
	Мы уже упоминали, что постоянная функция $f(n) = 1$ мультипликативна, как и функция тождества $f(n) = n$. Так как $\tau$ и $\sigma$ могут быть представлены в виде:
	$$\tau(n) = \sum \limits_{d\mid n} 1 \text{\quad и\quad} \sigma(n) =\sum \limits_{d\mid n} d$$
	указанный результат непосредственно вытекает из теоремы 6-4.\\[1cm]
    \end{proof}

	\begin{center}
    \LARGE {\textsf {\textbf {ПРОБЛЕМЫ 6.1}}}\\[5mm]
   	\end{center}
   	
\begin{enumerate}
	\item Пусть $m$ и $n$ - положительные целые числа, а $p_{1},p_{2},\ldots,p_{r}$ $p$ - различные простые числа, которые делят хотя бы одно из $m$ или $n$. Тогда $m$ и $n$ могут быть записаны в виде:
	
	$${m = p_{1}^{k_{1}}p_{2}^{k_{2}}\ldots p_{r}^{k_{r}},\quad \text{где } k_{i}\ge0 \text{ для } i = 1,2,\ldots,r}$$ 
	$${n = p_{1}^{j_{1}}p_{2}^{j_{2}}\ldots p_{r}^{j_{r}},\quad \text{где } j_{i}\ge0 \text{ для } i = 1,2,\ldots,r}$$
	
	Докажим это:
	
	\setlength{\parskip}{1ex}
	
	$${gcd (m,n) = p_{1}^{u_{1}}p_{2}^{u_{2}}\ldots p_{r}^{u_{r}},\quad lcm(m,n) = p_{1}^{\upsilon_{1}}p_{2}^{\upsilon_{2}}\ldots p_{r}^{\upsilon_{r}}},$$
	
	\setlength{\parskip}{1ex}
		
	где $u_{i} = min \{k_{i},j_{i}\}$, меньшее из $ k_{i} $ и $ j_{i} $; а $\upsilon_{i} = max \{k_{i},j_{i}\}$, большее из $ k_{i} $ и $ j_{i} $.
	
	
	\item Используйте Проблему 1 для расчета gcd (12378, 3054) и lcm (12378, 3054).
	
	\item Выведите из Проблемы 1, что gcd $(m, n)$ lem $(m, n)$ = $mn$ для для натуральных чисел $m$ и $n$.
	
	\item В обозначениях Проблемы 1 покажите, что gcd $(m, n)$ = 1 тогда и только тогда, когда $k_{i}j_{i} = 0$, для $i = 1,2,\ldots,r$.
	 
	\item
	\begin{enumerate} 
	    \item Убедитесь, что $\tau(n) = \tau(n+1) = \tau(n+2) = \tau(n+3)$ для $n$ = 3655 и 4503.
        \item Когда $n$ = 14, 206 и 957, покажите, что $\sigma(n) = \sigma(n+1)$.
    \end{enumerate}
    \item Для любого целого числа $n \ge 1$ установите неравенство $\tau(n)\le 2\sqrt{n}$. [{\itshape Подсказка}: если
    $d\mid n$, то один из $d$ или меньше или равен $\sqrt{n}$.]
    \item Докажите, что:
    \begin{enumerate} 
    	\item $\tau(n)$ является нечетным целым числом тогда и только тогда, когда $n$ - полный квадрат;
    	\item $\sigma(n)$ является нечетным целым числом тогда и только тогда, когда $n$ является полным квадратом или дважды полным квадратом. [{\itshape Подсказка}: если $p$ нечетное простое число, то $1+p+p^{2}+\ldots+p^{k}$ нечетно, только когда $k$ четное.]
    \end{enumerate}
    \item Покажите, что $\sum_{d\mid n}$ $1/d=\sigma(n)/n$ для каждого натурального числа $n$.
    \item Если $n$ - целое число без квадратов, докажите, что $\tau(n) = 2^{r}$, где $r$ - число простых делителей $n$.
    \item Установите следующие утверждения: 
    \begin{enumerate} 
    	\item Если $n = p_{1}^{k_{1}}p_{2}^{k_{2}}\ldots p_{r}^{k_{r}}$ простой множитель от $n > 1$, то
    	$$1 > \frac{n}{\sigma(n)} > \left (1 - \frac{1}{p_{1}}\right)\left (1 - \frac{1}{p_{2}}\right)\ldots \left (1 - \frac{1}{p_{r}}\right).$$
    	\item Для любого натурального числа $n$,  $\sigma(n!)/n!\ge1+1/2+1/3+\ldots+1/n.$
    	
    	[{\itshape Подсказка}: см. Проблему 8.]
    	\item  Если $n > 1$ является составным числом, то $\sigma(n)>n + \sqrt{n}$. [{\itshape Подсказка}: Пусть $d\mid n$, где $1 < d < n$, поэтому $1 < n\mid d < n$. Если $d \le \sqrt{n}$, то $n\mid d \ge \sqrt{n}$.]
    \end{enumerate}
    \item Учитывая положительное целое число $k > 1$, покажите, что существует бесконечно много целых чисел $n$, для которых $\tau(n)=k$, но не более, чем $n$ при $\sigma(n)=k$. [{\itshape Подсказка}: используйте Проблему 10(а).]
    \item 
    \begin{enumerate} 
    	\item Найдите форму всех натуральных чисел $n$, удовлетворяющих $\tau(n) = 10$. Какое наименьшее натуральное число, для которого это верно?
    	\item Покажите, что нет натуральных чисел $n$, удовлетворяющих $\sigma(n) = 10$. [{\itshape Подсказка}: обратите внимание, что для $n > 1$, $\sigma(n) > n$.]
    \end{enumerate}
    \item Докажите, что существует бесконечно много пар целых чисел $m$ и $n$ при $\sigma(m^{2}) = \sigma(n^{2}).$ [{\itshape Подсказка}: выберите $k$ так, чтобы gcd$(k,d) = 1$, и рассмотрите целые числа $m = 5k$, $n = 4k$.]
    \item Для $k\ge 2 $, покажите каждое из следующего:
    \begin{enumerate} 
    	\item $n = 2^{k-1}$ удовлетворяет уравнению $\sigma(n) = 2n - 1$;
    	\item если $2^{k}-1$ простое, то $n = 2^{k-1}\left(2^{k}-1\right)$ удовлетворяет уравнению $\sigma(n) = 2n$;
    	\item если $2^{k} - 3$ простое, то $n = 2^{k-1}\left(2^{k}-1\right)$ удовлетворяет уравнению $\sigma(n) = 2n+2$. 
    \end{enumerate}
    	Не известно, существуют ли целые числа $n$, для которых $\sigma(n) = 2n + 1$.
    	\item Если $n$ и $n+2$ - два простых числа, установить, что $\sigma (n+2) = \sigma (n)+2$; это также верно для $n$ = 434 и 8575.
    	\item 
    	\begin{enumerate} 
    		\item Для любого целого числа $n > 1$ докажите, что существуют целые числа $n_{1}$ и $n_{2}$, причем $\tau(n_{1})+\tau(n_{2}) = n$.
    		\item Докажите, что из гипотезы Гольдбаха следует, что для каждого четного целого числа $2n$ существуют целые числа $n_{1}$ и $n_{2}$, причем $\sigma (n_{1}) + \sigma (n_{2}) = 2n $.
    	\end{enumerate}
        \item Для фиксированного целого числа $k$ покажите, что функция $f$, определенная как
        $f(n)=n^{k}$, мультипликативна.
        \item Пусть $f$ и $g$ - мультипликативные функции, такие что $f(p^{k})=g(p^{k}) $ для каждого простого числа $p$ и $k\ge 1.$ Докажите, что $f = g$.
        \item Докажите, что если $f$ и $g$ являются мультипликативными функциями, то и их произведение $fg$ и частное $f/g$ (всякий раз, когда последняя функция определена).
        \item Определим функцию $p$, принимая $p(1)=1$ и $p(n)=2^{r}$, если простые множители $n > 1$ является $n = p_{1}^{k_{1}}p_{2}^{k_{2}}\ldots p_{r}^{k_{r}}$. Например, $p(8) = 2$ и $p(10) = p(36) = 2^{2}$.
        \begin{enumerate} 
        	\item Выведите, что $p$ является мультипликативной функцией.
        	\item Найти формулу для $F(n) = \sum_{d\mid n}p(d)$ в условии из простой факторизации из n.
        \end{enumerate}
        \item Для любого положительного целого числа $n$ докажите, что $\sum_{d\mid n} \tau(d)^{3} = \left(\sum_{d\mid n}\tau(d)\right)^{2}$. [{\itshape Подсказка}: обе стороны рассматриваемого уравнения являются мультипликативными функциями от $n$, так что достаточно рассмотреть случай $n=p^{k}$, где $p$ - простое число.]
        \item Учитывая $n\ge 0$ пусть $\sigma_{s}(n)$ обозначает сумму степеней $s$th положительных делителей $n$; то есть, $$\sigma_{s}(n)=\sum_{d\mid n}d^{s}.$$
        
        Проверить следующее:
        \begin{enumerate} 
        	\item $\sigma_{0} = \tau$ и $\sigma_{1}=\sigma$.
        	\item $\sigma_{s}$ - это мультипликативная функция. [{\itshape Подсказка}: функция $f$, определяемая $f(n)=n^{s}$, является мультипликативной.]
        	\item Если $n = p_{1}^{k_{1}}p_{2}^{k_{2}}\ldots p_{r}^{k_{r}}$ является простой факторизацией $n$, то $$\sigma_{s}(n)=\left(\frac{p_{1}^{s(k_{1}+1)-1}}{p_{1}^{s}-1}\right)\left(\frac{p_{2}^{s(k_{2}+1)-1}}{p_{2}^{s}-1}\right)\ldots \left(\frac{p_{r}^{s(k_{r}+1)-1}}{p_{r}^{s}-1}\right).$$
        \end{enumerate}
        \item Для любого положительного целого числа $n$ покажите, что
        \begin{enumerate} 
        	\item $\sum \limits_{d\mid n}\sigma(d)=\sum \limits_{d\mid n}\dfrac{n}{d}\tau(d)$, и
        	\setlength{\parskip}{2mm}
        	\item $\sum \limits_{d\mid n}\dfrac{n}{d}\sigma(d)= \sum \limits_{d\mid n}d\tau(d)$
        	
        	[{\itshape Подсказка}: Поскольку функции $F(n)=\sum _{d\mid n}\sigma(d)$ и $G(n)=\sum _{d\mid n}n/d \tau(d) $ являются мультипликативными, достаточно доказать, что $F(p^{k})=G(p^{k})$ для любого простого числа $p$.]\\[5mm]
        \end{enumerate}
\end{enumerate}
 
        \begin{flushleft}
    	\LARGE {\textsf {\textbf {6.2 Формула инверсии Мебиуса}}}
        \end{flushleft}
        Введем еще одну естественно определенную функцию на положительных целых числах, $\mu $ - функцию Мебиуса.
        
        \begin{center}
    	Определение 6-3. Для положительного целого числа $n$ определите $\mu $ по правилам 
        \end{center}
    	\[
    	\mu(n) =
    	\begin{cases}
    	 1,  \text{ если $n=1$} \\
    	 0,  \text{ если $p^{2}\mid n$ для некоторый простых $p$}\\
    	 (-1)^{r},  \text{ если $n=p_{1}p_{2}\ldots p_{r}$, где $p_{i}$ являются различными простыми числами.}
    	\end{cases}
    	\]
    	
    	Иными словами, Определение 6-3 гласит, что $\mu(n)=0$, если $n$ не является целым числом без квадратов, в то время как $\mu(n)=(-1)^{r}$, если $n$ не является квадратом с $r$ простыми множителями. Например: $\mu(30)=\mu(2\cdot3\cdot5)=(-1)^{3}=-1$. Первые несколько значений $\mu$ таковы:\\[1mm] $$\mu(1)=1,\mu(2)=-1,\mu(3)=-1,\mu(4)=0,\mu(5)=-1,\mu(6)=1,\ldots$$\\[1mm]
    	Если $p$-простое число, то ясно, что $\mu(p) = - 1$; также $\mu(p^{k}) = 0$ для $k \ge 2$.
    	
    	Как читатель, возможно, уже догадался, $\mu$ - функция Мебиуса является
    	мультипликативной. Это содержание\\
    	
    	{\setlength{\leftskip}{9mm}
    		\setlength{\rightskip}{9mm}
         \noindent Теорема 6-5. {\itshape Функция u является мультипликативной функцией.}\\[5mm]
        {\itshape Доказательство}: Мы хотим показать, что $\mu(mn)=\mu(m)\mu(n)$, когда $m$ и $n$ относительно просты. Если $p^{2}\mid m$ или $p^{2}\mid n$, $p$ простое, то $p^{2}\mid mn$; следовательно,
        $\mu(mn)=0=\mu(m)\mu(n) $, и формула выполняется тривиально. Поэтому мы можем предположить, что и $m$, и $n$ являются целыми числами без квадрата. Скажем, $m = p_{1}p_{2}\ldots p_{r}$, $n = q_{1}q_{2}\ldots q_{r}$, числа $p$ и $q$ разные и простые. Затем 
        \begin{align*}
          \mu(mn)=\mu(p_{1}\ldots p_{r}q_{1}\ldots q_{r})&=(-1)^{r+s}\\
          &=(-1)^{r}(-1)^{s}=\mu(m)\mu(n),
        \end{align*}
        что завершает доказательство.\\
    
        }
    
        Давайте посмотрим, что произойдет, если вычислить $\mu(d)$ для всех положительных делителей $d$ целого числа $n$ и добавить результаты. В случае $n = 1$, ответ прост; здесь, $$\sum \limits_{d\mid n}\mu(d)=\mu(1)=1.$$
        Предположим, что $n > 1$ и тогда: $$F(n)=\sum \limits_{d\mid n}\mu(d).$$ 
        Чтобы подготовить почву, мы сначала вычисляем $F(n)$ для мощности простого числа, скажем $n=p^{k}$. Положительными делителями являются просто $k + 1$ целых чисел $1,p,p^{2},\ldots,p^{k}$, так что
        \begin{align*}
          F(p^{k})=\sum \limits_{d\mid p^{k}}\mu(d)&=\mu(1)+\mu(p)+\mu(p^{2})+\ldots+\mu(p^{k})\\
          &= \mu(1)+\mu(p)=1+(-1)=0.
        \end{align*}	
\end{document}