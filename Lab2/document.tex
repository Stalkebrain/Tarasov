\documentclass{beamer}% тип документа

% далее идёт преамбула
\usepackage[utf8]{inputenc}
\usepackage[russian]{babel}
\usetheme{Boadilla}

\title{Пример презентации в \LaTeX}
\subtitle{При Использовании \textbf{Beamer}}
\author{Тарасов Егор Александрович\\
	Компьютерная безопастность I}
\institute{Институт физико-математических наук и информационных технологий БФУ им. И. Канта}
\date{\today}

\begin{document}% начало презентации
	
	\begin{frame}
		\titlepage
	\end{frame}
	
	\begin{frame}
		\frametitle{План}
		\begin{center}
			Модуль: Учебная практика(КБ-1,2020).
		\end{center}
		\begin{enumerate}
			\item Задачи практики:
			\begin{enumerate}
				\item распределенные системы управления версиями на примере GIT.
				\item система верстки LaTeX.
			\end{enumerate} 
			\item Методы решения:
			\begin{enumerate}
				\item создайть репозиторий на любом из бесплатных веб-сервисов.
				\item в репозитории создать две папки: ''Lab 1'' , ''Lab 2''.
				\item установить TeXstudio.
				\item выполнять лабораторные работы по инструкции на сайте курса.
			\end{enumerate} 
		\end{enumerate} 
	\end{frame}
	
	\begin{frame}
		\frametitle{Результаты}
		\begin{center}
			В результате прохождения учебной практики(КБ-1,2020),я значительно расширил свои знания, а, в частности, ознакомился c системой управления версиями на примере \textbf{GIT}, а так же научился работать с системой верстки \textbf{\LaTeX}.
		\end{center}
	\end{frame}
\end{document}